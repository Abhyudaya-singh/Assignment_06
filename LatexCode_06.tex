\documentclass[12pt]{article}
\usepackage[english]{babel}
\usepackage{natbib}
\usepackage{url}
\usepackage[utf8x]{inputenc}
\usepackage{amsmath}
\usepackage{graphicx}
\graphicspath{{images/}}
\usepackage{parskip}
\usepackage{fancyhdr}
\usepackage{vmargin}
\setmarginsrb{3 cm}{2.5 cm}{3 cm}{2.5 cm}{1 cm}{1.5 cm}{1 cm}{1.5 cm}

\title{5 Solutions to Covid-19 provided by BioMedical Engineers}					
\author{21111004}								
\date{4 MARCH 2022}								
\makeatletter
\let\thetitle\@title
\let\theauthor\@author
\let\thedate\@date
\makeatother

\pagestyle{fancy}
\fancyhf{}
\rhead{\theauthor}
\lhead{\thetitle}
\cfoot{\thepage}

\begin{document}
\begin{titlepage}
	\centering
    \includegraphics[scale = 0.20]{logo.jpg}\\[1.0 cm]	
    \textsc{\LARGE National Institute Of Technology \newline\\\\ RAIPUR}\\[2.0 CM]
    
	\textsc{\Large ASSIGNMENT 06}\\[0.5 cm]				% Course Code
	\rule{\linewidth}{0.4 mm} \\[0.4 cm]
	{ \huge \bfseries \thetitle}\\
	\rule{\linewidth}{0.4 mm} \\[1.5 cm]
	
	\begin{minipage}{0.6\textwidth}
		\begin{flushleft} \large
			\emph{Submitted To:}\\
			Saurabh Gupta\\
            Department Of Basic Biomedical Engineering\\
			\end{flushleft}
			\end{minipage}~
			\begin{minipage}{0.4\textwidth}
            
			\begin{flushright} \large
			\emph{Submitted By :}\\
			Abhyudaya Kumar Singh\\
            21111004\\
		\end{flushright}
        
	\end{minipage}\\[2 cm]
\end{titlepage}

\tableofcontents
\pagebreak

\section{Introduction}
COVID-19 is one of the most severe global health crises that humanity has ever faced. Researchers have restlessly focused on developing solutions for monitoring and tracing the viral culprit, SARS-CoV-2, as vital steps to break the chain of infection. Even though biomedical engineering (BME) is considered a rising field of medical sciences, it has demonstrated its pivotal role in nurturing the maturation of COVID-19 diagnostic technologies. Within a very short period of time, BME research applied to COVID-19 diagnosis has advanced with ever-increasing knowledge and inventions, especially in adapting available virus detection technologies into clinical practice and exploiting the power of interdisciplinary research to design novel diagnostic tools or improve the detection efficiency. \newline
Following are the few solutions provided by biomedical engineers dring covid 19 pandemic.


\section{Continuous Positive Airway Pressure (CPAP)}
Continuous Positive Airway Pressure (CPAP) which is initially intended to prevent airways collapse in sleep apnoea patients, but has been shown to be beneficial to COVID patients if applied early enough in the progression of the disease.\newline
A well-fitted face mask is an essential component of a CPAP system and as such it is quite intrusive. CPAP is only appropriate for patients who are capable of some breathing strength as CPAP effectively opposes some resistance to expiration. Variants exist that either automatically adjust the level of pressure to the patients breathing characteristics (APAP) or have different levels of pressure for inspiration and expiration (BiPAP). CPAP usually supplies (filtered) air to the patient but most masks have a port for supplementing the supply with oxygen.


\section{Ventilators}
Patients who cannot breathe spontaneously need to be put on a ventilator. Ventilators are capable of replacing the breath function and patients in an advanced state of respiratory distress are usually intubated and sedated at the beginning of the treatment.\newline
Ventilators are capable of replacing the breath function and patients in an advanced state of respiratory distress are usually intubated and sedated at the beginning of the treatment. They are complex systems providing the healthcare professionals with a lot of flexibility to adapt the assisted breathing settings and to be able to wean recovering patients off the ventilator gradually.

 \section{Patient Monitoring System}
An essential element of the ICU equipment is the monitoring equipment that keeps track of some of the patient vitals especially when they are ventilated and sedated but also during their recovery phase to ensure the regime of ventilation is optimised for their condition. \newline
Since covid is highly infectious it is very important to quarantine covid patients but at the same time doctors need to monitor health of covid patients too. With the increasing number of cases it is becoming difficult to keep a track on the health conditions of many quarantined patients. Ventilators already provide their set of patient parameters, but usually patient monitors are separate devices as they continue to be useful after the patient can resume breathing on their own unassisted.
\section{Oxygen Concentrators}
Oxygen concentrators represent an attractive alternative to tanks although this is not typically in use while caring for COVID-19 patients in hospital settings. Oxygen concentrators extract oxygen from the air on demand and supply it directly to the patient. Concentrators come in a variety of sizes from a portable shoulder bag form factor, to higher capacity stationary machines for patients who need oxygen 24/7.\newline
Variants of oxygen supply include high flow nasal oxygen (HFNO) which delivers warmed and humidified oxygen, to avoid the drying of airways, at high flow rates - typically tens of litres/min) at body temperature and up to 100% RH and 100% oxygen.

\section{Sourcing vital equipments}

When a pandemic like COVID-19 takes hold, biomedical engineers work around the clock to ensure the right systems, equipment and devices are in place. Their expertise becomes a vital link in coordinating supply chains of life-saving medical technology.\newline
 All instruments are under the constant supervision of a biomedical engineer. Before they even receive the complaint that the machine is not working, they check and resolve the issue.
\end{document}

